\section{Gabriel Filipowicz}

Piekne zdjecie Szopa Bartłomieja ( zobacz figure \ref{Pracz}):

\begin{figure}[h]
\centering
\includegraphics[width=0.7\textwidth]{pictures/szopito.jpg}
\caption{Czyż nie uroczy?}
\label{Pracz}
\end{figure}

\centering  Wzór obliczający położenie Szopa we wszechświecie
\begin{equation}
\centering
X^{1}_{szop} = s*z_{o}*\frac{p}{i}\ + \int_1^9 to \ 
\end{equation}

Nagle czas przyszedł na tabelę(przerażająca w jej osobie tabela \ref{kindness}):

\begin{table}[htbp]
\centering
\label{kindness}
\begin{tabular}{|l|l|l|l|}
\hline
aa  & bb  & cc  & dd    \\ \hline
15  & 60  & 240 & 960   \\ \hline
1.8 & 0.9 & 2   & 4.8   \\ \hline
dd  & ee  & ff  & gg    \\ \hline
\end{tabular}
\end{table}

\vspace{2cm}

Niezbędnik domowy:

\begin{itemize}
    \item Solniczka
    \item toster 
    \item ser żółty 
    \item żelazko 
\end{itemize}

Czego nie chcemy mieć w domu:

\begin{enumerate}
    \item karaluchy
    \item nielegalne substancje
    \item plecak innej osoby
\end{enumerate}
\vspace{2cm}

\begin{flushleft}
\textbf{Gra} to dowolna sytuacja \underline{konfliktowa}, gracz natomiast to dowolny jej uczestnik. 

\vspace{1cm}
\emph{Graczem może być na przykład człowiek}, przedsiębiorstwo lub zwierzę. Każda strona wybiera pewną strategię postępowania, po czym zależnie od strategii własnej oraz innych uczestników każdy gracz otrzymuje wypłatę w jednostkach użyteczności. 

\vspace{1cm}
\underline {Zależnie od gry jednostki te mogą reprezentować pieniądze}, wzrost szansy na przekazanie własnych genów czy też cokolwiek innego, z czystą satysfakcją włącznie. Wynikowi gry zwykle przyporządkowuje się pewną \textbf{wartość liczbową}.
\end{flushleft}
