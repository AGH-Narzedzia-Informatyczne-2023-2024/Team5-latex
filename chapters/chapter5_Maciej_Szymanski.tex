\newpage
\section{Maciej Szymański}

\vspace{1 cm}

Oto zdjęcie ze śmiesznym kotem (Spójrz na Figure~\ref{fig:Funny cat})
\begin{figure}[htbp]
    \centering
    \includegraphics[width=0.9\textwidth]{pictures/funny_cat.jpg}
    \caption{Śmieszny kot}
    \label{fig:Funny cat}
\end{figure}

Table~\ref{tab:taka tabela} wygląda tak:
\begin{table}[htbp]
\centering
\begin{tabular}{|c|c|c|}
\hline
\textbf{Kolumna I} & \textbf{Kolumna II} & \textbf{Kolumna III} \\ \hline
2839               & dwnj2               & jeije2               \\ \hline
23ui2              & 32uidw              & ndje3e               \\ \hline
3298e              & ej2ie               & jei2je               \\ \hline
\end{tabular}
\caption{taka tabela}
\label{tab:taka tabela}
\end{table}

Równanie Eulera:
\[e^{ \pm i\theta } = \cos \theta \pm i\sin \theta\]
\newpage
Lista nienumerowana:
\begin{itemize}
  \item Element I
  \item Element II
  \item Element III
\end{itemize}

\vspace{0.5 cm}

Lista numerowana:
\begin{enumerate}
  \item Element I
  \item Element II
  \item Element III
\end{enumerate}

\vspace{0.5 cm}

\paragraph{}
Ten tekst to \textbf{pierwszy} akapit. A to \underline{podkreślenie}. I jeszcze \textit{pochylenie}. Ten tekst to \textbf{pierwszy} akapit. A to \underline{podkreślenie}. I jeszcze \textit{pochylenie}. Ten tekst to \textbf{pierwszy} akapit. A to \underline{podkreślenie}. I jeszcze \textit{pochylenie}. Ten tekst to \textbf{pierwszy} akapit. A to \underline{podkreślenie}. I jeszcze \textit{pochylenie}. Ten tekst to \textbf{pierwszy} akapit. A to \underline{podkreślenie}. I jeszcze \textit{pochylenie}. Ten tekst to \textbf{pierwszy} akapit. A to \underline{podkreślenie}. I jeszcze \textit{pochylenie}. Ten tekst to \textbf{pierwszy} akapit. A to \underline{podkreślenie}. I jeszcze \textit{pochylenie}.
\paragraph{}
Ten tekst to drugi akapit.Ten tekst to drugi akapit.Ten tekst to drugi akapit.Ten tekst to drugi akapit.Ten tekst to drugi akapit.Ten tekst to drugi akapit.Ten tekst to drugi akapit.Ten tekst to drugi akapit.Ten tekst to drugi akapit.Ten tekst to drugi akapit.Ten tekst to drugi akapit.Ten tekst to drugi akapit.Ten tekst to drugi akapit.Ten tekst to drugi akapit.Ten tekst to drugi akapit.Ten tekst to drugi akapit.Ten tekst to drugi akapit.Ten tekst to drugi akapit.Ten tekst to drugi akapit.


