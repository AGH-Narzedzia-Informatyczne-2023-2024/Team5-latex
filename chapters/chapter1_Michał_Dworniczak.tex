\section{Michał Dworniczak}

Funkcja dzeta Riemanna: $$\zeta(z) = \sum_{n = 1}^{\infty}\left(\frac{1}{n}\right)^z$$


Zdjecie:

\begin{figure}[htbp]
    \centering
    \includegraphics[width=0.5\textwidth]{pictures/vernonroche.jpg}
    \caption{Vernon Roche}
    \label{fig:vernon}
\end{figure}

Tabela:

\begin{table}[htbp]
\centering
\def\arraystretch{1.2}
\begin{tabular}{|c|c|c|c|c|c|}
	\hline
	$x$ & 1 & 2 & 3 & 4 & 5\\ \hline
	$f(x)$ & $\frac{1}{2}$ & 11 & 12 & 13 & 14\\ \hline
\end{tabular}
\caption{Wartości funkcji f(x) dla danego x}
\label{tab:funkcja}
\end{table}

\vspace{2cm}

Przykładowa lista numerowana

\begin{enumerate}
    \item \LaTeX
    \item GitHub
    \item Python
\end{enumerate}

\vspace{0.5cm}

Przykładowa lista nienumerowana

\begin{itemize}
    \item Przykładowa
    \item lista
    \item nienumerowana
\end{itemize}

\vspace{0.5cm}

\begin{center}
    \textbf{Goryl} – rodzaj ssaków naczelnych z podrodziny Homininae w obrebie rodziny człowiekowatych (Hominidae). Preferuja naziemny tryb życia, sa zwierzetami roślinożernymi, zamieszkuja lasy tropikalne w Afryce. Dzielone sa na dwa gatunki i według nadal trwajacej debaty (2007) do czterech lub pieciu podgatunków.\par
\end{center}
\begin{flushleft}
    Prowadza spokojny tryb życia, spedzajac wiekszość czasu w grupach rodzinnych, które przemieszczaja sie po terytorium o powierzchni 5–30 km². Noc spedzaja w legowisku z liści i gałezi. Żeruja rano i po południu, odpoczywajac w nocy i około południa. Tryb życia i dieta różnia sie u poszczególnych podgatunków. Goryle potrafia wykorzystywać kamienie jako proste narzedzia służace do rozbijania łupin orzechów.
\end{flushleft}
\begin{flushright}
    Zdjecie numer \ref{fig:vernon} \textbf{NIE} przedstawia goryla.\par
    Tabela numer \ref{tab:funkcja} nie ma żadnego 
    zwiazku z gorylami.
\end{flushright}


